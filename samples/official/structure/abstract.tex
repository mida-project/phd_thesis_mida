% #############################################################################
% Abstract
% !TEX root = main.tex
% #############################################################################

\vspace*{-7.50mm}%

As intelligent agents advance, they promise to enhance decision-making in high-stakes domains.
This thesis focuses on designing and adapting these agents for specific audiences like radiology clinicians.
It explores prerequisites for integrating anthropomorphic intelligent agents as second-reader diagnostic support, their role in clinical workflows, user acceptance dynamics, and the impact of personalized recommendations.

The initial phase explores the adoption of intelligent agents by proposing a model rooted in the unified theory of acceptance and use of technology.
This model examines factors influencing agents' adoption throughout the medical imaging workflow, emphasizing the role of security, risk, and trust.
These findings provide insights into improving the integration of intelligent agents into healthcare settings.

We then explore applying deep learning to medical imaging diagnostics, utilizing a human-centric design approach, focusing on breast cancer.
The development and evaluation of the {\it BreastScreening-AI} framework, integrating multimodal imaging and AI for image analysis, demonstrate gains in diagnostic efficiency and reduced cognitive workload with improvements in clinicians' satisfaction.
The thesis underscores the transformative power of intelligent agents in augmenting clinicians' workflows and decision-making.

The dissertation concludes by exploring customized communication between intelligent agents and clinicians.
We investigated the influence of two contrasting communication tones on clinicians' receptivity and performance:
(1) suggestive (non-assertive); and
(2) imposing (assertive).
Findings reveal that assertiveness-based agents can reduce medical errors and enhance satisfaction, introducing a novel perspective on adaptive communication in assisted healthcare.

Ultimately, this dissertation contributes to human-computer interaction by emphasizing clinicians' needs and the essentials of incorporating intelligent agents into healthcare.
The execution of design interventions emphasizes the importance of a human-centered approach, focusing on clinicians' needs in developing novel, accessible solutions in critical domains.
The thesis investigates clinicians' attitudes towards intelligent agents, providing insights to direct future healthcare design and research, marking it as a crucial field reference.

\vfil

\noindent
{\bf Key-words:} Human-Computer Interaction, User-Centered Design, Artificial Intelligence, Intelligent Agents, Medical Imaging