% #############################################################################
% Resumo
% !TEX root = main.tex
% #############################################################################

\vspace*{-7.50mm}%

\`{A} medida que os agentes inteligentes avan\c{c}am rapidamente, estes prometem melhorar significativamente a tomada de decis\~{o}es em dom\'{i}nios de alto risco.
Esta tese concentra-se no aspecto menos explorado de projetar e adaptar esses agentes para p\'{u}blicos espec\'{i}ficos, como cl\'{i}nicos de radiologia.
Aprofunda-se nos pr\'{e}-requisitos para a integra\c{c}\~{a}o bem-sucedida de agentes antropom\'{o}rficos como suporte diagn\'{o}stico de segundo leitor, explorando o seu papel nos fluxos de trabalho cl\'{i}nicos, nas din\^{a}micas de aceita\c{c}\~{a}o pelos utilizadores e no impacto das recomenda\c{c}\~{o}es personalizadas.

Numa fase inicial, exploramos a ado\c{c}\~{a}o de agentes inteligentes com um modelo baseado na {\it unified theory of acceptance and use of technology}.
A disserta\c{c}\~{a}o examina fatores cr\'{i}ticos, como seguran\c{c}a, risco e confian\c{c}a, que influenciam a ado\c{c}\~{a}o dos agentes nos fluxos de trabalho com imagens m\'{e}dicas.
Essas descobertas s\~{a}o valiosas para a integra\c{c}\~{a}o bem-sucedida de agentes inteligentes na sa\'{u}de.

Exploramos a aplica\c{c}\~{a}o dos m\'{e}todos de {\it deep learning} no diagn\'{o}stico do cancro de mama, seguindo uma abordagem centrada no utilizador, como grupos de foco, observa\c{c}\~{o}es e entrevistas.
O desenvolvimento e avalia\c{c}\~{a}o de uma {\it framework} chamada \textit{BreastScreening-AI}, que integra t\'{e}cnicas de imagens multimodais e intelig\^{e}ncia artificial, demonstrando ganhos significativos na efici\^{e}ncia do diagn\'{o}stico e redu\c{c}\~{a}o da carga cognitiva, melhorando a satisfa\c{c}\~{a}o dos cl\'{i}nicos.
A tese destaca o poder transformador desses agentes na melhoria dos fluxos de trabalho e tomada de decis\~{o}es dos cl\'{i}nicos.

A disserta\c{c}\~{a}o conclui explorando a comunica\c{c}\~{a}o personalizada entre agentes inteligentes e cl\'{i}nicos.
Para isso, investigamos a influ\^{e}ncia de dois tons de comunica\c{c}\~{a}o contrastantes na receptividade e no desempenho dos cl\'{i}nicos:
(1) sugestivo (n\~{a}o assertivo); e
(2) impositivo (assertivo).
As descobertas revelam que agentes baseados na assertividade podem reduzir erros m\'{e}dicos e aumentar a satisfa\c{c}\~{a}o, introduzindo uma nova perspectiva no design da comunica\c{c}\~{a}o adaptativa em assist\^{e}ncia \`{a} sa\'{u}de.

Esta disserta\c{c}\~{a}o contribui para o campo da intera\c{c}\~{a}o humano-computador ao enfatizar as necessidades dos cl\'{i}nicos e a incorpora\c{c}\~{a}o de agentes inteligentes na sa\'{u}de.
A implementa\c{c}\~{a}o de interven\c{c}\~{o}es de design destaca a import\^{a}ncia de uma abordagem centrada no utilizador, desenvolvendo solu\c{c}\~{o}es inovadoras, acess\'{i}veis e f\'{a}ceis de usar em dom\'{i}nios cr\'{i}ticos.
A tese investiga as atitudes dos cl\'{i}nicos em rela\c{c}\~{a}o aos agentes inteligentes, fornecendo {\it insights} para futuros trabalhos e pesquisas em sa\'{u}de, sendo uma refer\^{e}ncia crucial nesta \'{a}rea.

\vfil

\noindent
{\bf Palavras-chave:} Interação Pessoa-Máquina, Conceção Centrada no Utilizador, Inteligência Artificial, Agentes Inteligentes, Imagens Médicas

\clearpage