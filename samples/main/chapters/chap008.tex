% #############################################################################
% This is Chapter 8
% !TEX root = main.tex
% #############################################################################
% Change the Name of the Chapter i the following line
\fancychapter{Conclusion}
\clearpage
% The following line allows to ref this chapter
\label{chap:chap008}

This chapter presents the concluding findings of our dissertation, focusing on integrating \ac{AI} into clinical workflows.
It examines factors influencing \ac{AI} acceptance, emphasizing the importance of a human-centered approach for integration in high-stakes domains.
The chapter explores design principles achieved in this thesis, emphasizing practical implementation in intelligent agents.
It synthesizes our research journey, highlighting achievements and paving the way for \ac{AI} integration in healthcare.

\section{Findings Overview}
\label{sec:chap008001}

Our research encompassed three key areas:
(1) acceptance and adoption of \ac{AI} systems in the clinical workflow;
(2) human-centered design approach in radiology; and
(3) personalization of \ac{AI} recommendations for clinicians.
In the acceptance and adoption study, trust emerged as a critical factor influencing clinicians' acceptance of \ac{AI}-based assistance in medical imaging diagnosis.
Clinicians' demographic characteristics, such as gender, age, and educational background, moderated the relationship between trust, security, and risk, and their intention to use \ac{AI} systems.
Integrating \ac{HCI} principles into developing \ac{AI}-based assistance enhanced usability and effectiveness, improving clinical workflows and patient outcomes.
These findings highlight the importance of considering clinicians' needs and characteristics when designing intelligent agents.

In the application of a human-centered design approach, our proposed design interventions played a pivotal role in the development of the intelligent agent for radiology.
By following a \ac{UCD} process and incorporating feedback from clinicians, we were able to create an \ac{AI} system that effectively addressed their needs and preferences.
The design interventions ceased crucial elements such as explainability functionalities, visualizations, and control mechanisms, aimed at facilitating accurate diagnosis and fostering clinicians' understanding and trust in \ac{AI} recommendations.
These interventions were crucial in optimizing the workflow, reducing \acp{FP}, \acp{FN}, and diagnostic time.
Ultimately, improving the accuracy of patient classification.
The success of the design interventions underscores the importance of a human-centered approach in designing \ac{AI} systems for real-world clinical settings.

Our investigation also focused on the personalization of explanations for human clinicians.
Through an experimental study comparing a conventional agent to an assertiveness-based agent, we demonstrated the merits of tailoring communication tone and providing granular explanations.
The assertiveness-based agent outperformed the conventional agent regarding diagnostic time and accuracy, particularly for novice clinicians.
Additionally, clinicians exhibited increased trust in the assertiveness-based agent, perceiving it as more reliable, competent, and understandable.
These findings highlight the significance of personalizing \ac{AI} systems to clinicians' expertise and experience, fostering higher engagement and acceptance.
The implications of this research extend beyond the medical domain, providing insights into the design of \ac{AI} systems in fields requiring personalized \acp{HAII}.

\section{Design Principles}
\label{sec:chap008002}

In this thesis, we leverage \ac{HCI} principles to optimize \ac{AI} system design and adoption in healthcare.
Through rigorous research, we identify crucial design principles for successful integration.
Our goal is to enhance \ac{UX}, promote acceptance, and improve effectiveness in healthcare.

\vspace{2.00mm}

\noindent
The following design principles emerged from our study:

\vspace{0.05mm}

\begin{enumerate}

\item {\bf Clinician-Centric Design}:
Prioritize the unique needs, characteristics, and workflows of clinicians in the design of \ac{AI}-based assistance systems for decision-making in real-world healthcare settings.
Incorporate their input and feedback throughout the design process to ensure the system meets their requirements and preferences.

\vspace{0.05mm}

\item {\bf Explainability and Transparency}:
Enhance clinicians' understanding and trust in \ac{AI} recommendations by incorporating explainability functionalities, visualizations, and control mechanisms.
Provide clear and interpretable granular explanations of the system's reasoning and decision-making process, allowing clinicians to comprehend and validate each \ac{DL} output.

\vspace{0.05mm}

\item {\bf Personalized Communication}:
Customize the communication style and approach of \ac{DL} models to address the specific needs and expertise levels of different user groups.
Tailor the presentation of information, \ac{AI} guidance, and recommendations to align with the professional experience and knowledge of clinicians, promoting engagement and acceptance.

\vspace{0.05mm}

\item {\bf Usability and Workflow Integration}:
Design \ac{AI} systems that seamlessly integrate into the existing clinical workflow, minimizing disruptions and enhancing efficiency.
Ensure the system is intuitive, user-friendly, and easy to navigate, enabling clinicians to access and utilize \ac{AI} assistance seamlessly during their routine tasks without interruptions or changing their workflow.

\vspace{0.05mm}

\item {\bf Continuous Evaluation and Iterative Design}:
Establish mechanisms for ongoing evaluation of \ac{AI} systems' performance and user feedback.
Continuously gather data on system usability, accuracy, and impact on clinical outcomes to identify areas for improvement and refine the system iteratively.

\vspace{0.05mm}

\item {\bf Ethical Considerations and Regulatory Compliance}:
Adhere to ethical guidelines and regulatory requirements in the development and deployment of \ac{AI}-based systems in healthcare.
Safeguard patient privacy and confidentiality of each country or region.
Ensure the responsible use of \ac{AI} technology, and comply with relevant regional regulations to maintain trust and integrity.

\end{enumerate}

To summarize, we emphasize the importance of incorporating well-defined design principles to optimize \ac{AI} systems in healthcare.
Aligning these principles with the specific needs and requirements of clinicians enhances \ac{UX} and ensures the effectiveness of the systems.
This approach facilitates seamless integration of \ac{AI} technology into healthcare, improving outcomes for clinicians and patients.

\section{Summarizing Research Objectives}
\label{sec:chap008003}

This thesis started by investigating the acceptance and adoption of \ac{AI} systems in medical imaging diagnosis.
The research objectives (Section~\ref{sec:chap001002002}) covered clinicians' acceptance and adoption of \ac{AI}, emphasizing security, risk, and trust.
While exploring these factors, the research strived to understand better how \ac{AI} could be effectively integrated into healthcare, particularly in medical imaging diagnosis.

Subsequently, we incorporated \ac{HCI} principles into the design process, aiming to enhance usability and acceptance through a \ac{UCD} approach.
We employed focus groups, workshops, interviews, and observations to refine functionalities and visualizations of explanations iteratively.
This systematic introduction of design interventions tailored \ac{AI} recommendations to clinician needs.
Positive feedback guided us towards an optimal \ac{AI} solution, strengthening clinicians' understanding and confidence.
This revision significantly promoted clinicians' satisfaction and acceptance of \ac{AI} assistance in radiology, positively impacting the overall medical workflow.

The final objectives focused on the influence of assertiveness-based agents on personalized communication and the effects on medical assessment efficiency and efficacy, considering both novice and expert clinicians.
Our research follows several distinct aims.
Firstly, we aim to enhance our comprehension of how intelligent agents can be effectively incorporated into healthcare.
Secondly, our research objective was to discern the tangible effects of this integration.
Ultimately, we seek to enhance the efficiency, accuracy, and overall quality of clinical decision-making.

\section{Final Remarks}
\label{sec:chap008004}

As we wrap up, it is crucial to highlight the wide-ranging contributions made by this thesis.
This work encompasses several important areas, ranging from the examination of \ac{AI} acceptance and adoption in clinical procedures to the integration of \ac{HCI} principles in system design and the personalization of \ac{AI} recommendations.
These combined endeavors have significantly advanced the integration of \ac{AI} into healthcare, particularly in the context of medical imaging diagnosis.
Our findings provide meaningful insights and robust recommendations for designing, implementing, and evaluating \ac{AI}-based systems that are user-centric, effective, and well-suited to the unique needs of users in high-stakes domains.

Looking forward, we anticipate that the principles and insights gleaned from this dissertation will be crucial in guiding future studies and developments in the realm of \ac{AI} in healthcare.
The journey towards fully integrated, trusted, and valuable \ac{AI} systems in healthcare is complex and fraught with challenges.
However, with a clear understanding of user needs, a commitment to iterative and transparent design, and a focus on personalization, we believe the potential benefits for clinicians and patients alike are immense and worth the effort.
Our research marks a significant step in this direction, and we hope it serves as a valuable foundation for subsequent explorations in this burgeoning field.