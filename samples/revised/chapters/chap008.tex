% #############################################################################
% This is Chapter 8
% !TEX root = main.tex
% #############################################################################
% Change the Name of the Chapter i the following line
\fancychapter{Conclusion}
\clearpage
% The following line allows to ref this chapter
\label{chap:chap008}

This chapter presents the concluding findings of our dissertation, focusing on integrating \ac{AI} into clinical workflows.
It examines factors influencing \ac{AI} acceptance, emphasizing the importance of a human-centered approach for integration in high-stakes domains.
The chapter explores design principles achieved in this thesis, emphasizing practical implementation in intelligent agents.
It synthesizes our research journey, highlighting achievements and paving the way for \ac{AI} integration in healthcare.

\section{Findings Overview}
\label{sec:chap008001}

\textcolor{revised}{In our exploration, we focused on three critical topics:
(1) analyzing clinician acceptance and the integration of \ac{AI} into clinical workflows, highlighting the complexities involved in this transition;
(2) advocating for a human-centered design approach in radiology, emphasizing alignment with clinicians' needs to ensure usability; and
(3) fine-tuning \ac{AI} recommendations, attuning them to individual clinicians' decision-making styles.
Trust emerged as a crucial element affecting clinicians' receptivity towards \ac{AI}-assisted tools in medical imaging diagnostics.
Users' demographic characteristics, including gender, age, and educational background, subtly shaped their perceptions of trust, security, and risk associated with \ac{AI} systems, guiding their readiness to adopt these technologies in high-stakes domains.}

\textcolor{revised}{Our research was firmly grounded in \acs{UCD}, guiding a transformation in designing intelligent agents for radiology.
We prioritized clinicians' needs, behaviors, and feedback, actively involving them in all stages of system development.
Clinicians played a central role from initial concept to final implementation, ensuring functionalities like refined explainability, intuitive visualizations, and robust control mechanisms harmonized with clinical practices.
Embracing a reflective participatory design process, we established an ongoing feedback loop, enabling clinicians to evolve the system.
This iterative approach significantly improved the system's capabilities, leading to a remarkable reduction in \acp{FP} (approximately 22\%), \acp{FN} (about 4\%), resulting in a total of 26\% fewer medical errors, along with a fourfold decrease in diagnostic timelines.
These enhancements culminated in heightened accuracy in patient classification (up to 95\%), and the seamless integration of the system into clinical workflows.}

\textcolor{revised}{In our research, we explored customizing explanations for human clinicians through an empirical investigation comparing a conventional agent to an assertiveness-driven counterpart.
This experiment underscored the advantages of calibrating communication tone and delivering detailed explanations.
The assertiveness-based agent outperformed the conventional, benefiting novice clinicians with superior diagnostic efficiency and accuracy.
Furthermore, clinicians exhibited heightened trust in the assertiveness-based agent, seeing it as more dependable and proficient.
These outcomes underscore the importance of aligning \ac{AI} systems with clinicians' expertise and familiarity, fostering higher engagement and acceptance.
The implications of this research extend beyond the medical domain, providing valuable insights into designing \ac{AI} systems in various domains requiring personalized \acp{HAII}.}

\section{Design Principles}
\label{sec:chap008002}

\textcolor{revised}{In this thesis, we apply \ac{HCI} principles to enhance the design and adoption of intelligent agents in medical imaging.
Based on our design recommendations (Section~\ref{sec:chap007003}), we identify essential design principles for integrating \ac{AI} systems in healthcare settings.
Our primary goal is to improve \ac{UX}, foster acceptance, and enhance the effectiveness of intelligent agents in these high-stakes domains.}

\vspace{1.00mm}

\noindent
\textcolor{revised}{The following design principles have emerged from our comprehensive study:}

\vspace{0.05mm}

\begin{enumerate}

\item \textcolor{revised}{{\bf Clinician-Centric Design}:
Prioritize clinicians' unique requirements, attributes, and workflows when crafting \acs{AI}-driven decision-making tools for real-world healthcare.
Incorporate their insights and feedback throughout the design process to align the system with their needs and preferences.}

\vspace{0.05mm}

\item \textcolor{revised}{{\bf Explainability and Transparency}:
Enhance clinicians' trust in \ac{AI} recommendations through granular explainability, visual representations, and control mechanisms.
Provide clear insights into the system's rationale and decision-making process, enabling clinicians to validate each \ac{DL} output.}

\vspace{0.05mm}

\item \textcolor{revised}{{\bf Personalized Communication}:
Tailor the communication style and approach of \ac{DL} methods to diverse user groups' requirements and expertise levels.
Customize the presentation of information, \ac{AI} guidance, and recommendations to resonate with clinicians' backgrounds and knowledge, fostering engagement and acceptance.}

\vspace{0.05mm}

\item \textcolor{revised}{{\bf Usability and Workflow Integration}:
Design \ac{AI} systems that seamlessly assimilate into the clinical workflow, minimizing disruptions while bolstering efficiency.
Ensure the system's intuitiveness, user-friendliness, and ease of navigation, enabling clinicians to access and leverage \ac{AI} assistance within their routine tasks without workflow alterations.}

\vspace{0.05mm}

\item \textcolor{revised}{{\bf Continuous Evaluation and Iterative Design}:
Establish mechanisms for ongoing assessment of \ac{AI} system performance and user feedback.
Continually gather data on system usability, accuracy, and impact on clinical outcomes.
This iterative approach identifies areas for enhancement and allows for continuous system refinement.}

\vspace{0.05mm}

\item \textcolor{revised}{{\bf Ethical Considerations and Regulatory Compliance}:
Adhere to ethical standards and regulatory prerequisites during the development and deployment of \ac{AI}-based healthcare systems.
Safeguard patient privacy and confidentiality in compliance with local laws and regulations.
Ensure responsible \ac{AI} technology use and comply with regional regulations to maintain trust and integrity.}

\end{enumerate}

\textcolor{revised}{In summary, we emphasize the critical role of well-defined design principles in optimizing \ac{AI} systems for healthcare.
Aligning these principles with clinicians' needs enhances the \ac{UX} and ensures system effectiveness.
This approach facilitates seamless \ac{AI} integration into healthcare, ultimately improving outcomes for clinicians and patients.
Next, we outline our research objectives.}

\section{Summarizing Research Objectives}
\label{sec:chap008003}

\textcolor{revised}{This thesis explored the acceptance and adoption of intelligent agents in medical imaging diagnosis, primarily focusing on breast cancer detection.
Research objectives delved into clinicians' acceptance and adoption of \ac{AI} (Section~\ref{sec:chap001002002} of Appendix~\ref{chap:app001}), emphasizing security, risk, and trust.
Concurrently, the research aimed to integrate \ac{AI} systems into healthcare, particularly in medical imaging diagnosis.
When examining previous numbers, the impact of our work becomes evident (Section~\ref{sec:app005018} of Appendix~\ref{chap:app005}).
Reducing radiologists' workload by approximately 48\% could significantly impact burnout rates (Section~\ref{sec:chap005006001}), potentially decreasing them to around 20.98\%~\cite{doi:10.1148/radiol.212631}.
Additionally, our research reveals a substantial 26\% reduction in medical errors in breast cancer diagnosis, potentially saving thousands of lives yearly~\cite{doi:10.3322/caac.21492}.
Clinicians can now diagnose patients four times faster, potentially cutting diagnosis time by up to 2.25 weeks per patient~\cite{WAYMEL2019327}, leading to markedly improved patient outcomes.}

\textcolor{revised}{Subsequently, this study integrated \ac{HCI} principles into the design process, employing a \ac{UCD} approach.
Qualitative research methods, including focus groups, workshops, interviews, and observations, iteratively refined \ac{AI} functionalities, and explanatory visualizations.
\ac{AI} recommendations were tailored to clinicians' needs, effectively reducing errors and enhancing diagnostic accuracy.
The research objectives align seamlessly with these impactful outcomes, seeking to comprehend \ac{AI} integration and its effects, ultimately elevating efficiency, accuracy, and clinical decision quality, as evidenced by substantial reductions in medical errors and diagnostic time.}

\section{Final Remarks}
\label{sec:chap008004}

\textcolor{revised}{In conclusion, this thesis has made substantial contributions across several critical areas.
These include examining the acceptance and adoption of intelligent agents in clinical settings, the infusion of \ac{HCI} principles into system design, and the personalization of \ac{DL} predictions.
These collective efforts have significantly advanced the integration of \ac{AI} recommendations into the healthcare domain, particularly in the context of medical imaging diagnosis.
Our findings offer valuable insights and robust suggestions for creating, implementing, and assessing \acs{AI}-based systems that prioritize user-centricity, effectiveness, and alignment with the unique needs of high-stakes domains.}

\textcolor{revised}{Looking ahead, we anticipate that the principles and insights from this dissertation will play a pivotal role in shaping future studies and developments of \ac{AI} systems in healthcare.
The path toward fully integrated, trusted, and impactful intelligent agents in real-world clinical settings is complex and challenging.
Nonetheless, armed with a deep comprehension of user requirements, a commitment to iterative and transparent design practices, and a focus on personalization.
We believe the potential advantages for clinicians and patients warrant substantial efforts.
Our research represents a significant stride in this direction, aspiring to serve as a solid foundation for subsequent inquiries in this burgeoning field.}