% #############################################################################
% This is Chapter 2
% !TEX root = main.tex
% #############################################################################
% Change the Name of the Chapter i the following line
\fancychapter{Background}
\clearpage
% The following line allows to ref this chapter
\label{chap:chap002}

Chapter~\ref{chap:chap002} provides valuable background information about the breast cancer domain in the context of this thesis.
While it does not present any original contributions, it can be omitted by readers already familiar with the subject matter.
For more comprehensive details, including additional background information and medical specifications, please refer to Appendix~\ref{chap:app001}.

\section{Medical Imaging Diagnosis}
\label{sec:chap002001}

Screening aims to identify cancer at earlier stages of several diseases when treatment can be more successful~\cite{McKinney2020}.
However, two main difficulties arise.
First, the amount of data to be processed has been increasing significantly and dramatically surpasses the throughput capabilities of radiologists~\cite{HANNA20181709}.
Second, processing such an amount of data in a timely fashion without compromising the reliability of the diagnosis is very challenging~\cite{McKinney2020}.
These difficulties have been the driving force behind the recent trend in radiomics\footnotemark[4] (Section~\ref{sec:chap002006}) and the integration of \ac{AI} techniques into medical imaging~\cite{doi:10.1148/radiol.2015151169}.
For a more detailed exploration of these challenges, please refer to Section~\ref{sec:app001001} in Appendix~\ref{chap:app001}.

%%%%%%%%%%%%%%%%%%%%%%%%%%%%%%%%%%%%%%%%%%%%%%%%%%%
\footnotetext[4]{Involving the high-throughput extraction of quantitative imaging features, {\it radiomics} intent of creating mining information from radiological images. Large-scale data sharing is necessary for the validation and full potential that {\it radiomics} represents.}
%%%%%%%%%%%%%%%%%%%%%%%%%%%%%%%%%%%%%%%%%%%%%%%%%%%

\section{Breast Cancer Domain}
\label{sec:chap002002}

Multi-modal data is crucial in breast cancer diagnosis (Figure~\ref{fig:fig005} on Section~\ref{sec:app001002} of Appendix~\ref{chap:app001}), impacting the clinical workflow~\cite{10.1117/1.JBO.22.4.046008}.
While \ac{MG} is the primary screening modality, it may not suffice for accurate diagnoses, particularly in dense breasts where lesion detection is challenging~\cite{10.1093/jbi/wbaa010}.
Conversely, adipose breasts allow precise lesion visualization, necessitating additional modalities for complementing the diagnosis.
\ac{US} and \acs{DCE-MRI} modalities provide valuable information, facilitating lesion visualization.
Ultimately, the primary objective in breast cancer diagnosis is to classify tumors based on severity into benign or malign categories~\cite{SHAN2016980}.

\section{Severity Classification}
\label{sec:chap002003}

Clinical guidelines emphasize the importance of regular image screenings for evaluating the risk of breast cancer~\cite{MIAO201817}.
To enhance consistency in the interpretation of imaging findings, the \acs{ACR} introduced the \acs{BI-RADS}~\cite{d2018breast}, which provides a standardized process (Figure~\ref{fig:fig020} on Section~\ref{sec:app001003} of Appendix~\ref{chap:app001}) for describing imaging characteristics and a classification framework with six categories to assess the probability of malignancy.
In the present study, a \acs{DNN} was trained to perform \acs{BI-RADS} classification (Section~\ref{sec:app004003} of Appendix~\ref{chap:app004}).
While this section offers a concise overview of severity classification, Section~\ref{sec:app001003} of Appendix~\ref{chap:app001} delves into further details regarding breast lesion classification.

\section{Lesion Typification}
\label{sec:chap002004}

The lesion typification~\cite{doi:10.1148/radiol.2018181371} is also challenging in this thesis background.
The goal of this section is to explain what is the importance of the lesions, {\it i.e.}, masses (Section~\ref{sec:chap002004001}) and the microcalcifications (Section~\ref{sec:chap002004002}), that will feed the \ac{AI} algorithms (Chapter~\ref{chap:chap005} and Chapter~\ref{chap:chap006}) and provide visual information to clinicians.
For the severity classification~\cite{8611096, 9231684}, it is essential to know the type of mass shapes, microcalcification patterns, and size to understand their importance to the algorithms.
This section summarizes the importance of lesion typification, which will be further detailed in Section~\ref{sec:app001004} of Appendix~\ref{chap:app001}.

\subsection{Masses}
\label{sec:chap002004001}

Mass typification is a crucial aspect of a breast cancer diagnosis.
Benign masses commonly exhibit round, oval, or lobular shapes, while malignant masses are often characterized by lobular, irregular, or architectural distortion shapes (Figure~\ref{fig:fig021} on Section~\ref{sec:app001004001} of Appendix~\ref{chap:app001}).
Margins play a significant role in lesion evaluation.
A circumscribed margin indicates a benign lesion, whereas microlobulated, indistinct, and spiculated margins raise suspicion, with spiculated margins being the most concerning.
An obscured margin, where part of the margin is concealed by fibroglandular tissue, necessitates further investigation, such as an \ac{US} examination (Figure~\ref{fig:fig018}).
It is important to note that this section summarizes the topic, with further details to be presented in Section~\ref{sec:app001004001} of Appendix~\ref{chap:app001}.

\subsection{Microcalcifications}
\label{sec:chap002004002}

In this section, we provide an overview of microcalcification typification, covering five types and their corresponding patterns (Figure~\ref{fig:fig022} on Section~\ref{sec:app001004002} of Appendix~\ref{chap:app001}).
These patterns include diffuse, regional, group, linear, and segmental, each representing distinct characteristics of microcalcifications.
Understanding these patterns is crucial in diagnosing and assessing the severity of breast cancer.
Further details about microcalcification patterns will be discussed in Section~\ref{sec:app001004002} of Appendix~\ref{chap:app001}.
It's important to note that a comprehensive and accurate dataset for clinicians and \ac{AI} algorithms requires a multimodal approach beyond mammography (Figure~\ref{fig:fig018}).
The subsequent section (Section~\ref{sec:chap002005}) will explore the current imaging workflow and emphasize the significance of multi-modal data availability.

\section{Radiology Reading Room}
\label{sec:chap002005}

This section summarizes the intricate radiological workflow, further detailed in Section~\ref{sec:app001005} of Appendix~\ref{chap:app001}.
The radiological workflow consists of three stages:
(1) {\it examination};
(2) {\it diagnosis}; and
(3) {\it report}.
Understanding each stage is vital for both the \ac{HCI} and \ac{AI} communities, as it offers valuable insights into the advancements made in this thesis, particularly in assisting radiologists.

During the {\it examination} (Section~\ref{sec:app001005001}) stage, the radiologist reviews patient records and correlates medical imaging with other relevant information.
The {\it diagnosis} (Section~\ref{sec:app001005002}) stage involves analyzing medical images and utilizing various sources of information to enhance accuracy.
Finally, the {\it report} (Section~\ref{sec:app001005003}) stage captures the radiologist's findings and is a critical communication tool for other healthcare professionals.
In Section~\ref{sec:app001005004} of Appendix~\ref{chap:app001}, we also summarize the main clinical procedures involved in the radiological workflow.

\section{Radiomics}
\label{sec:chap002006}

\ac{DL} has emerged as a powerful asset in {\it radiomics}~\cite{litjens2017survey}, improving the accuracy of breast cancer diagnoses.
These \ac{DL} models are contributing significantly to the {\it radiomics} pipeline (Figure~\ref{fig:fig023} on Section~\ref{sec:app001006} of Appendix~\ref{chap:app001}), including image pre-processing, feature extraction, and classification~\cite{10.1007/978-3-030-59716-0_71}.
However, their `black box' nature, need for extensive annotated data, and data privacy concerns pose challenges~\cite{litjens2017survey}.
This thesis explores these challenges and solutions, emphasizing the role of \ac{HCI} in enhancing trust and adoption of \ac{AI}-assisted medical tools (Chapter~\ref{chap:chap004}).
Further insights into the role of \ac{DL} models in {\it radiomics} are provided in Section~\ref{sec:app001006} of Appendix~\ref{chap:app001}.

\section{Medical Imaging Standards}
\label{sec:chap002007}

This section summarizes the \acs{DICOM} format, a widely used medical standard for storing images~\cite{Trivedi2019}.
It supports various medical information types and file format conversions.
\acs{DICOM} files use associative arrays for hierarchical data structures (Figure~\ref{fig:fig025} on Section~\ref{sec:app001007} of Appendix~\ref{chap:app001}), with each key referred to as a \acs{DICOM} tag.
Standardizing these tags improves readability and facilitates the storage of exams and patient information (Section~\ref{sec:chap002008}).
Further details on this topic can be found in Section~\ref{sec:app001007} of Appendix~\ref{chap:app001}.

\section{Medical Imaging Storage}
\label{sec:chap002008}

The \acs{PACS}~\cite{carter2018digital} offers efficient storage and a simple way to access the \acs{DICOM} images in a variety of modalities.
The deployment of these technologies requires specific packages and environments, which will be further detailed in Section~\ref{sec:app001008} of Appendix~\ref{chap:app001}.
Our platform provides essential tools for deploying multimodality strategies, interactive image visualization/manipulation, and study navigation in a web browser.
This paves the road for integrating multimodality strategies into, \acs{PACS} as it only needs a web browser that is always accessible on clinical workstations.
Further details on deploying these technologies and environments can be found in Section~\ref{sec:app001008} of Appendix~\ref{chap:app001}.