\begin{table}[htbp]
\centering
\resizebox{\columnwidth}{!}{%
\begin{tabular}{|l|l|l|l|}
\hline
\multicolumn{4}{|c|}{AI Design Guidelines}                                                                    \\ \hline
G1 & Make clear what the system can do.                    & G10 & Scope services when in doubt.              \\ \hline
G2 & Make clear how well the system can do what it can do. & G11 & Make clear why the system did what it did. \\ \hline
G3 & Time services based on context.                       & G12 & Remember recent interactions.              \\ \hline
G4 & Show contextually relevant information.               & G13 & Learn from user behavior.                  \\ \hline
G5 & Match relevant social norms.                          & G14 & Update and adapt cautiously.               \\ \hline
G6 & Mitigate social biases.                               & G15 & Encourage granular feedback.               \\ \hline
G7 & Support efficient invocation.                         & G16 & Convey the consequences of user actions.   \\ \hline
G8 & Support efficient dismissal.                          & G17 & Provide global controls.                   \\ \hline
G9 & Support efficient correction.                         & G18 & Notify users about changes.                \\ \hline
\end{tabular}
}
\caption{The 18 human-AI interaction design guidelines~\cite{10.1145/3290605.3300233}. The guidelines were developed based on a comprehensive review of existing literature on human-AI interaction and were validated through multiple rounds of evaluation. The guidelines are intended to inform the design of AI systems that interact with human users, with a focus on promoting user trust, transparency, and control over the interaction. These guidelines were used to inform the design of the medical assistant discussed in this thesis.}
\label{tab:tab004}
\end{table}