\begin{table}[htbp]
\begin{tabular*}{\textwidth}{ c @{\extracolsep{\fill}} c c @{\extracolsep{\fill}} c c @{\extracolsep{\fill}} c c }
\toprule
\\
\small
&
\multicolumn{2}{ c }{Effort}
&
\multicolumn{2}{ c }{Performance}
&
\multicolumn{2}{ c }{Frustration}
\\
\cmidrule(lr){2-3}
\cmidrule(lr){4-5}
\cmidrule(lr){6-7}
Condition & F & Sig. & F & Sig. & F & Sig. \\
\\
\bottomrule
\\
Current & 0.534 & 0.661 & 5.556 & 0.003$\star$ & 2.392 & 0.082 \\
Assistant & 0.664 & 0.578 & 0.319 & 0.811 & 0.408 & 0.748 \\
\\
\bottomrule
\hfill
\end{tabular*}
\caption{ The ANOVA factorial analysis table regarding NASA-TLX for \textit{Effort (Eff.)}, \textit{Performance (Per.)} and \textit{Frustration (Fru.)}, where {\it F} is the variation between sample means and the variation within samples. To determine whether any of the differences between the means are statistically significant, the {\it Sig.} for significance was used. On the present study, a 20-point Likert Scale was used regarding Workload. The factorial analysis was described assuming $\alpha = 0.05$. Also, each time $p < 0.05$ it is marked with the $\star$ symbol.}
\label{tab:tab002}
\end{table}