% #############################################################################
% Abstract
% !TEX root = main.tex
% #############################################################################

An abstract in a PhD thesis is a succinct summary that captures the essence of the conducted research, designed to give readers a quick overview of the study's scope, methodology, results, and implications without needing to delve into the full document.
Positioned at the beginning, it serves as a standalone section, informing readers about the research's significance and outcomes.
It starts by introducing the research topic, setting the stage by outlining the background and the specific problem or question addressed, aiming to highlight the study's relevance within the field.
The abstract then succinctly states the main objectives or research questions, clarifying the study's direction and purpose.
The methodology section describes the research approaches and techniques, covering data collection and analysis methods to assure the reader of the study's rigor.
Key outcomes are summarized in the results section, pointing out significant data, discoveries, or insights and their contribution to the field.
Conclusions reflect on the findings' implications, including their relevance to existing knowledge or future research directions, encapsulating the thesis's scholarly contribution.
Keywords included enhance the thesis's discoverability for researchers searching on related topics.
As a critical component, the abstract must be clear and concise, effectively communicating the thesis's essence to the reader, typically within 250-500 words, setting the tone for the entire document.